%
% report.tex
%

\documentclass[a4paper, 12pt]{report}

% UTF-8 characters.
\usepackage[utf8]{inputenc}

% Page format.
\usepackage[top=3cm, bottom=3cm, left = 2cm, right = 2cm]{geometry}

% 1.2 line spacing.
\usepackage{setspace}
\spacing{1.2}

% Romanian language support.
\usepackage{fontspec}
\usepackage{polyglossia}
\setdefaultlanguage{romanian}
\SetLanguageKeys{romanian}{indentfirst=true}
\usepackage{csquotes}
\DeclareQuoteStyle{romanian}
  {\quotedblbase}
  {\textquotedblright}
  {\guillemotleft}
  {\guillemotright}

% Embed images.
\usepackage{graphicx}
\graphicspath{./assets/}

% Pretty URLs.
\usepackage[
    colorlinks,
    linkcolor={black},
    menucolor={black},
    citecolor={black},
    urlcolor={blue}
]{hyperref}

% No page breaks between chapters.
\usepackage{etoolbox}
\makeatletter
\patchcmd{\chapter}{\if@openright\cleardoublepage\else\clearpage\fi}{}{}{}
\makeatother

% Chapter number and title on the same line.
\usepackage{titlesec}
\titleformat{\chapter}[hang]
{\normalfont\huge\bfseries}{\thechapter.}{0.5em}{}

% Abstract on the same page.
\renewenvironment{abstract}
{\small\quotation
{\bfseries\noindent{\large\abstractname}\par\nobreak\smallskip}}
{\endquotation}

% Colors.
\usepackage[table]{xcolor}

% Better tables.
\usepackage{tabularx}

% Multiple sub-rows in tables.
\usepackage{multirow}

% Metadata and title.
\usepackage{titling}
\title{Raport privind cercetarea și analiza unui set de date despre filme}
\author{
  Mathe Armand Cezar
  \hspace{2cm}
  Tritean Sergiu Daniel
}
\date{Iunie 2023}
\renewcommand{\maketitle}{
  \begin{titlepage}
    \begin{center}
      \large
      \textbf{Universitatea Babeș-Bolyai} \\
      \vspace{0.5cm}
      \textbf{Facultatea de Științe Economice și Gestiunea Afacerilor}\\
      \vspace{0.5cm}
      \textbf{Specializarea Informatică Economică}

      \vspace{7cm}

      \Huge
      \thetitle

      \vspace{6cm}

      \large
      \theauthor

      \vspace{3.7cm}

      \thedate
    \end{center}
  \end{titlepage}
}

\usepackage{lipsum}

% Document.
\begin{document}

\maketitle

\pagebreak

\tableofcontents

\pagebreak

\chapter{Introducere}

Această lucrare își propune să cerceteze și să analizeze un set de date despre
filme, cu scopul de a determina dacă putem prezice popularitatea sau votul mediu
al unui film pe baza altor caracteristici. Acest lucru ar putea ajuta la
direcționarea investițiilor în domeniu spre proiectele care ar putea avea un
succes mai mare.

Setul de date ales este TMDB 15000 Movies Dataset, provenit de pe platforma
Kaggle
\footnote{\url{https://www.kaggle.com/datasets/prabhakarz/tmdb-15000-movies-dataset-with-credits}}
. Acesta conține informații despre filmele cu cele mai mari rating-uri din
toate timpurile și include caracteristici precum limba originală a filmului,
popularitate, data lansării, votul mediu și numărul de voturi ale fiecărui film.
În plus, sunt disponibile informații referitoare la genuri, cuvinte cheie și
membrii distribuției și echipei de producție.

Pentru a analiza acest set de date am folosit Python (împreună cu librăriile
numpy, pandas, scikit-learn, matplotlib, seaborn, și xgboost) în mediile
interactive Jupyter Notebooks.

Principalele întrebări de cercetare adresate în această lucrare sunt:

\begin{enumerate}
  \item În ce măsură putem prezice popularitatea unui film pe baza altor
        caracteristici?
  \item În ce măsură putem prezice votul mediu al unui film pe baza altor
        caracteristici?
\end{enumerate}

În continuare vom examina setul de date pentru a determina relațiile dintre
caracteristicile sale și pentru a crea modele care să prezică popularitatea
sau votul mediu al fiecărui film.

\chapter{Setul de date}

Setul de date a fost inițial colectat prin scraping de pe site-ul TMDB
(The Movie Database), o resursă populară pentru informații despre filme și
seriale TV. Acest set de date a fost ulterior disponibilizat pe Kaggle, o
platformă pentru învățarea automată și analiza datelor, de către utilizatorul
Prabhakar Pal
\footnote{\url{https://www.kaggle.com/prabhakarz}}
care a efectuat procesul de colectare a datelor prin intermediul API-ului
TMDB.

Setul de date include aproximativ 15.000 de filme și conține o serie de
caracteristici pentru fiecare film, cum ar fi:

\begin{itemize}
  \item \textbf{Limba originală a filmului}: Limba în care a fost produs filmul.
        Contine codul ISO 639-1 pentru limba respectivă.
  \item \textbf{Titlul original filmului}: Numele filmului în limba originală a
        acestuia.
  \item \textbf{Privire de ansamblu}: O scurtă descriere a filmului.
  \item \textbf{Data de lansare}: Anul în care filmul a fost lansat.
  \item \textbf{Număr de voturi}: Numărul de voturi primite de film.
  \item \textbf{Votul mediu}: Votul mediu acordat de utilizatori filmului.
  \item \textbf{Popularitate}: Scorul de popularitate al filmului.
  \item \textbf{Genuri}: Lista genurilor la care este asociat filmul.
  \item \textbf{Cuvinte cheie}: Lista cuvintelor cheie asociate filmului.
  \item \textbf{Actori}: Lista actorilor care au jucat în film.
  \item \textbf{Echipa de producție}: Lista membrilor echipei de producție.
\end{itemize}

\chapter{Rezultate si discuții}

In acest capitol vom prezenta rezultatele obtinute in urma cercetarii si
analizei setului de date despre filme. Vom examina fiecare intrebare de
cercetare in parte si vom discuta metodele de analiza utilizate, precum si
rezultatele obtinute.

\section{Intrebarea de cercetare 1: In ce masura putem prezice popularitatea unui film pe baza altor caracteristici?}

Pentru a raspunde la aceasta intrebare am utilizat mai multe modele de regresie:
linear regression, decision tree regression, random forest regression, gradient
boost regression si XGBoost regression. Am incercat sa construim modele care sa
ia in considerare diferite combinatii de caracteristici ale filmelor.

INSERT CONTENT - Prezentarea detaliata a modelelor testate si a rezultatelor
obtinute

In final, cel mai bun model pentru a prezice popularitatea unui film s-a dovedit
a fi Gradient Boosting Regression, cu o radacina medie patratica a erorii (RMSE)
de 1.120490.

\section{Intrebarea de cercetare 2: In ce masura putem prezice votul mediu al unui film pe baza altor caracteristici?}

La fel ca in cazul primei intrebari de cercetare, am utilizat mai multe modele
pentru a incerca sa prezicem votul mediu al unui film: linear regression,
decision tree regression, random forest regression, gradient boost regression si
XGBoost regression. Am luat in considerare diferite combinatii de caracteristici
ale filmelor in construirea modelelor.

INSERT CONTENT - Prezentarea detaliata a modelelor testate si a rezultatelor
obtinute

In final, cel mai bun model pentru a prezice votul mediu al unui film s-a
dovedit a fi Gradient Boosting Regression, cu o RMSE de 0.098926.

\section{Discutii}

Rezultatele obtinute indica faptul ca este posibil sa prezicem popularitatea
unui film sau votul mediu al acestuia pe baza altor caracteristici. Cu toate
acestea, trebuie sa ne amintim ca exista limite ale cercetarii noastre. De
 exemplu, setul de date utilizat poate fi incomplet sau nu reprezentativ pentru
 industria cinematografica in ansamblu. De asemenea, este posibil ca alte
 caracteristici ale filmelor care nu sunt incluse in acest set de date sa joace
 un rol important in determinarea succesului lor.

Astfel, pentru imbunatatirea cercetarii noastre si a preciziei predictiilor
facute prin modelele noastre, ar putea fi necesara adaugarea altor
caracteristici ale filmelor si extinderea setului de date utilizat.

INSERT TABLE - Tabel cu rezumatul tuturor modelelor testate

INSERT FIGURE - Grafic ce ilustreaza relatia dintre diferite caracteristici ale
filmelor (vote\_average, vote\_count, popularity si release\_year)

Legenda:

G - Genuri, K20 - Cuvinte cheie (top 20), P - Popularitate, V - Numar de voturi

\begin{table}
  \centering
  \renewcommand{\arraystretch}{2}

  \begin{tabularx}{\textwidth}{|ccccc|ccccc|}
    \cline{0-9}
    \multicolumn{5}{|c|}{\textbf{Parametrii de intrare}} & \multicolumn{5}{c|}{\textbf{Modele de regresie}} \\
    \cline{0-9}
    % Parameters.
    \multicolumn{1}{|c}{\textbf{G}}
    & \multicolumn{1}{c}{\textbf{K20}}
    & \multicolumn{1}{c}{\textbf{K50}}
    & \multicolumn{1}{c}{\textbf{P}}
    & \multicolumn{1}{c|}{\textbf{V}}
    % Models.
    & \multicolumn{1}{c}{\textbf{L}}
    & \multicolumn{1}{c}{\textbf{DT}}
    & \multicolumn{1}{c}{\textbf{RF}}
    & \multicolumn{1}{c}{\textbf{GB}}
    & \multicolumn{1}{c|}{\textbf{XGB}} \\
    \cline{0-9}
    X & X & & X & X
    & \cellcolor{green!25}0.177730
    & \cellcolor{yellow!25}0.133896
    & \cellcolor{yellow!25}0.106875
    & \cellcolor{yellow!25}0.098926
    & \cellcolor{yellow!25}0.104490 \\
    \cline{0-9}
    X & X & & X & X
    & \cellcolor{green!25}0.177730
    & \cellcolor{yellow!25}0.133896
    & \cellcolor{yellow!25}0.106875
    & \cellcolor{yellow!25}0.098926
    & \cellcolor{yellow!25}0.104490 \\
    \cline{0-9}
    X & X & & X & X
    & \cellcolor{green!25}0.177730
    & \cellcolor{yellow!25}0.133896
    & \cellcolor{yellow!25}0.106875
    & \cellcolor{yellow!25}0.098926
    & \cellcolor{yellow!25}0.104490 \\
    \cline{0-9}
    X & X & & X & X
    & \cellcolor{green!25}0.177730
    & \cellcolor{yellow!25}0.133896
    & \cellcolor{yellow!25}0.106875
    & \cellcolor{yellow!25}0.098926
    & \cellcolor{yellow!25}0.104490 \\
    \cline{0-9}
    X & X & & X & X
    & \cellcolor{green!25}0.177730
    & \cellcolor{yellow!25}0.133896
    & \cellcolor{yellow!25}0.106875
    & \cellcolor{yellow!25}0.098926
    & \cellcolor{yellow!25}0.104490 \\
    \cline{0-9}
    X & X & & X & X
    & \cellcolor{green!25}0.177730
    & \cellcolor{yellow!25}0.133896
    & \cellcolor{yellow!25}0.106875
    & \cellcolor{yellow!25}0.098926
    & \cellcolor{yellow!25}0.104490 \\
    \cline{0-9}
    X & X & & X & X
    & \cellcolor{green!25}0.177730
    & \cellcolor{yellow!25}0.133896
    & \cellcolor{yellow!25}0.106875
    & \cellcolor{yellow!25}0.098926
    & \cellcolor{yellow!25}0.104490 \\
    \cline{0-9}
    X & X & & X & X
    & \cellcolor{green!25}0.177730
    & \cellcolor{yellow!25}0.133896
    & \cellcolor{yellow!25}0.106875
    & \cellcolor{yellow!25}0.098926
    & \cellcolor{yellow!25}0.104490 \\
    \cline{0-9}
  \end{tabularx}

  \caption{Performanta modelelor}
  \label{tab:regression-performance}
\end{table}

% \section{Privire de ansamblu asupra datelor}



% \begin{figure}
%   \centering
%   \includegraphics[width=0.8\textwidth]{assets/numeric_columns.png}
%   \caption{Distribuția caracteristicilor numerice}
% \end{figure}

% \section{Analiza rezultatelor}

% În această secțiune, vom examina rezultatele analizei noastre asupra setului de
% date de filme. Vom aborda diferite metrici pentru a evalua performanța modelelor
% noastre și a înțelege semnificația acestor rezultate.

% \section{Interpretarea rezultatelor}

% Aici vom interpreta și explica rezultatele pe care le-am obținut în urma
% procesului de analiză. Vom încerca să identificăm modele, tendințe sau anomalii
% care pot oferi o înțelegere mai profundă a datelor noastre.

% \section{Implicații și discuții}

% Această secțiune va oferi un spațiu pentru discuții mai generale legate de
% rezultatele noastre. Vom încerca să evidențiem implicațiile acestora, precum și
% orice limitări ale analizei noastre.

% \section{Impactul asupra industriei cinematografice}

% În acest subcapitol, ne vom concentra asupra implicațiilor potențiale ale
% rezultatelor noastre asupra industriei cinematografice. În ce fel informațiile
% pe care le-am descoperit pot influența deciziile viitoare în cadrul industriei?

% \section{Limitări și viitoare direcții de cercetare}

% În final, vom aborda limitările prezentei analize și vom propune potențiale
% direcții pentru cercetările viitoare. Aceasta ne va permite să înțelegem mai
% bine unde putem îmbunătăți și ce alte aspecte ar putea fi explorate în viitor.

\chapter{Concluzia}

Aceasta este secțiunea în care vom trage concluziile finale în urma analizei
noastre asupra setului de date de filme. Vom sublinia cele mai importante
descoperiri și vom discuta semnificația lor în contextul mai larg al industriei
cinematografice.

\end{document}
