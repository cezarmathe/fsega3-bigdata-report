%
% report.tex
%

\documentclass[a4paper, 12pt]{report}

% UTF-8 characters.
\usepackage[utf8]{inputenc}

% Page format.
\usepackage[top=3cm, bottom=3cm, left = 2cm, right = 2cm]{geometry}

% 1.2 line spacing.
\usepackage{setspace}
\spacing{1.2}

% Romanian language support.
\usepackage{fontspec}
\usepackage{polyglossia}
\setdefaultlanguage{romanian}
\SetLanguageKeys{romanian}{indentfirst=true}
\usepackage{csquotes}
\DeclareQuoteStyle{romanian}
  {\quotedblbase}
  {\textquotedblright}
  {\guillemotleft}
  {\guillemotright}

% Embed images.
\usepackage{graphicx}
\graphicspath{./assets/}

% Pretty URLs.
\usepackage[
    colorlinks,
    linkcolor={black},
    menucolor={black},
    citecolor={black},
    urlcolor={blue}
]{hyperref}

% No page breaks between chapters.
\usepackage{etoolbox}
\makeatletter
\patchcmd{\chapter}{\if@openright\cleardoublepage\else\clearpage\fi}{}{}{}
\makeatother

% Chapter number and title on the same line.
\usepackage{titlesec}
\titleformat{\chapter}[hang]
{\normalfont\huge\bfseries}{\thechapter.}{0.5em}{}

% Abstract on the same page.
\renewenvironment{abstract}
{\small\quotation
{\bfseries\noindent{\large\abstractname}\par\nobreak\smallskip}}
{\endquotation}

% Colors.
\usepackage[table]{xcolor}

% Better tables.
\usepackage{tabularx}

% Multiple sub-rows in tables.
\usepackage{multirow}

% Metadata and title.
\usepackage{titling}
\title{Raport privind cercetarea și analiza unui set de date despre filme}
\author{
  Mathe Armand Cezar
  \hspace{2cm}
  Tritean Sergiu Daniel
}
\date{Iunie 2023}
\renewcommand{\maketitle}{
  \begin{titlepage}
    \begin{center}
      \large
      \textbf{Universitatea Babeș-Bolyai} \\
      \vspace{0.5cm}
      \textbf{Facultatea de Științe Economice și Gestiunea Afacerilor}\\
      \vspace{0.5cm}
      \textbf{Specializarea Informatică Economică}

      \vspace{7cm}

      \Huge
      \thetitle

      \vspace{6cm}

      \large
      \theauthor

      \vspace{3.7cm}

      \thedate
    \end{center}
  \end{titlepage}
}

\usepackage{lipsum}

% Document.
\begin{document}

\maketitle

\pagebreak

\tableofcontents

\pagebreak

\chapter{Introducere} \label{chapter:introducere}

Această lucrare își propune să cerceteze și să analizeze un set de date despre
filme, cu scopul de a determina dacă putem prezice popularitatea sau votul mediu
al unui film pe baza altor caracteristici. Acest lucru ar putea ajuta la
direcționarea investițiilor în domeniu spre proiectele care ar putea avea un
succes mai mare.

Setul de date ales este TMDB 15000 Movies Dataset, provenit de pe platforma
Kaggle
\footnote{\url{https://www.kaggle.com/datasets/prabhakarz/tmdb-15000-movies-dataset-with-credits}}
. Acesta conține informații despre filmele cu cele mai mari rating-uri din
toate timpurile și include caracteristici precum limba originală a filmului,
popularitate, data lansării, votul mediu și numărul de voturi ale fiecărui film.
În plus, sunt disponibile informații referitoare la genuri, cuvinte cheie și
membrii distribuției și echipei de producție.

Pentru a analiza acest set de date am folosit Python (împreună cu librăriile
numpy, pandas, scikit-learn, matplotlib, seaborn, și xgboost) în mediile
interactive Jupyter Notebooks.

Principalele întrebări de cercetare adresate în această lucrare sunt:

\begin{enumerate}
  \item În ce măsură putem prezice popularitatea unui film pe baza altor
        caracteristici?
  \item În ce măsură putem prezice votul mediu al unui film pe baza altor
        caracteristici?
\end{enumerate}

În continuare vom examina setul de date pentru a determina relațiile dintre
caracteristicile sale și pentru a crea modele care să prezică popularitatea
sau votul mediu al fiecărui film.

\chapter{Setul de date}

Setul de date a fost inițial colectat prin scraping de pe site-ul TMDB
(The Movie Database), o resursă populară pentru informații despre filme și
seriale TV. Acest set de date a fost ulterior disponibilizat pe Kaggle, o
platformă pentru învățarea automată și analiza datelor, de către utilizatorul
Prabhakar Pal
\footnote{\url{https://www.kaggle.com/prabhakarz}}
care a efectuat procesul de colectare a datelor prin intermediul API-ului
TMDB.

Setul de date include aproximativ 15.000 de filme și conține o serie de
caracteristici pentru fiecare film, cum ar fi:

\begin{itemize}
  \item \textbf{Limba originală a filmului}: Limba în care a fost produs filmul.
        Contine codul ISO 639-1 pentru limba respectivă.
  \item \textbf{Titlul original filmului}: Numele filmului în limba originală a
        acestuia.
  \item \textbf{Privire de ansamblu}: O scurtă descriere a filmului.
  \item \textbf{Data de lansare}: Anul în care filmul a fost lansat.
  \item \textbf{Număr de voturi}: Numărul de voturi primite de film.
  \item \textbf{Votul mediu}: Votul mediu acordat de utilizatori filmului.
  \item \textbf{Popularitate}: Scorul de popularitate al filmului.
  \item \textbf{Genuri}: Lista genurilor la care este asociat filmul.
  \item \textbf{Cuvinte cheie}: Lista cuvintelor cheie asociate filmului.
  \item \textbf{Actori}: Lista actorilor care au jucat în film.
  \item \textbf{Echipa de producție}: Lista membrilor echipei de producție.
\end{itemize}

\chapter{Rezultate si discuții}

Aceast capitol cuprinde o evaluare meticuloasă a setului de date referitor la
filme, precum și descrierea transformărilor care au fost aplicate acestuia. În
același timp, sunt furnizate răspunsuri la întrebările de cercetare, formulate
anterior în capitolul \hyperref[chapter:introducere]{\textit{Introducere}}.

\section{Analiza și transformarea Setului de Date}

Setul de date include câteva coloane cu valori numerice care s-ar putea dovedi
relevante pentru investigația noastră: \textbf{vote\_average} (denumit în cadrul
lucrării ca \textbf{Va}), \textbf{vote\_count} (referit în lucrare ca
\textbf{Vc}) și \textbf{popularity} (abreviat în lucrare ca \textbf{P}). În
analiza acestor coloane cu valori numerice am inclus și \textbf{release\_date}
cu scopul de a încerca să identificăm potențiale corelații.

Valoarea coloanei \textbf{vote\_average} se situează în intervalul de la 0 la 10
și prezintă o deviație standard redusă (2.555917). Am normalizat această coloană
utilizând metoda min-max, astfel încât valorile sale să se încadreze între 0 și
1.

Coloana \textbf{vote\_count} are valoarea maximă (33495) de 2 ordine de
magnitudine mai mare decât valoarea mediană (292), iar deviația standard
(2366.743281) este de asemenea semnificativă. Am aplicat logaritmarea în baza 10
urmată de standardizarea bazată pe scorul Z pentru a aduce valorile coloanei
într-un interval adecvat pentru analiză.

Coloana \textbf{popularity}, conform descrierii, se încadrează în intervalul de
la 0 la 100. Prin examinarea datelor, am observat că, deși majoritatea valorilor
sunt cuprinse în acest interval, există cazuri care depășesc limita superioară -
deviația standard (114.493917) fiind semnificativă. Pentru această coloană am
aplicat de asemenea standardizarea în funcție de scorul Z.

\begin{figure}
  \centering

  \includegraphics[width=0.8\textwidth]{assets/numeric_columns.png}

  \caption{Privire de ansamblu asupra coloanelor numerice, după transformări}
  \label{fig:numeric_columns}
\end{figure}

O vedere de ansamblu a acestor coloane, dupa transformări, se poate observa în
Figura \ref{fig:numeric_columns}.

\begin{figure}
  \centering

  \includegraphics[width=0.65\textwidth]{assets/boxplot_genres_popularity.png}

  \caption{Relația dintre genuri (\textbf{G}) și popularitate (\textbf{P})}
  \label{fig:genres_p}
\end{figure}

\begin{figure}
  \centering

  \includegraphics[width=0.65\textwidth]{assets/boxplot_genres_vote_average.png}

  \caption{Relația dintre genuri (\textbf{G}) și votul mediu (\textbf{Va})}
  \label{fig:genres_va}
\end{figure}

\begin{figure}
  \centering

  \includegraphics[width=0.65\textwidth]{assets/boxplot_keywords_popularity.png}

  \caption{Relația dintre cuvinte-cheie (\textbf{K20}) și popularitate (\textbf{P})}
  \label{fig:keywords_p}
\end{figure}

\begin{figure}
  \centering

  \includegraphics[width=0.65\textwidth]{assets/boxplot_keywords_vote_average.png}

  \caption{Relația dintre cuvinte-cheie (\textbf{K20}) și votul mediu (\textbf{Va})}
  \label{fig:keywords_va}
\end{figure}

În procesul analitic vom avea în vedere și două coloane pentru categorizare:
\textbf{genres} și \textbf{keywords}. Acestea cuprind informații referitoare la
genurile și cuvintele-cheie asociate fiecărui film. În vederea unei
analize eficiente a acestor coloane am recurs la implementarea unei
transformări de tip one-hot encoding, astfel încât am obținut o coloană
distinctă pentru fiecare gen sau cuvânt-cheie. În total, am generat 19 coloane
pentru genuri și peste 100 de coloane pentru cuvinte-cheie. Corelația dintre
genuri și popularitate poate fi vizualizată în Figura \ref{fig:genres_p}, în
timp ce relația dintre genuri și votul mediu este ilustrată în Figura
\ref{fig:genres_va}. Corelația dintre cuvinte-cheie și popularitate este
evidențiată în Figura \ref{fig:keywords_p}, iar relația dintre cuvinte-cheie și
votul mediu poate fi observată în Figura \ref{fig:keywords_va}.

\clearpage

\section{
  Gradul de predictibilitate a popularității unui film în funcție de alte
  caracteristici
}

În dorința de a stabili măsura în care putem prezice popularitatea unui film pe
baza altor atribute, am optat pentru experimentarea diverselor combinații de
parametri de intrare și modele de regresie. Pentru fiecare asociere de parametri
de intrare și model de regresie, am calculat eroarea pătratică medie (RMSE -
Root Mean Square Error) în vederea determinării nivelului de adecvare a datelor
la model. Rezultatele sunt prezentate, în ordinea performanței, în Tabela
\ref{tab:models_performance_p}.

Parametrii de intrare sunt:

\begin{itemize}
  \item \textbf{G} - Genurile asociate filmului
  \item \textbf{K20} - Cuvintele-cheie (top 20)
  \item \textbf{K50} - Cuvintele-cheie (top 50)
  \item \textbf{K100} - Cuvintele-cheie (top 100)
  \item \textbf{Va} - Evaluarea medie a filmului
  \item \textbf{Vc} - Numărul de voturi ale filmului
\end{itemize}

Pe baza evaluării efectuate am ajuns la concluzia că popularitatea unui film nu
poate fi prezisă cu precizie pe baza parametrilor de intrare menționați
anterior. Valoarea RMSE se situează mai aproape de 1 decât de 0 (de fapt, este
chiar peste 1), iar popularitatea a fost standardizată la o scală restrânsă,
fapt ce a condus la această concluzie.

\section{
  Determinarea gradului de predictibilitate a evaluării medii a unui film în
  funcție de alte caracteristici
}

În demersul nostru de a evalua măsura în care putem anticipa evaluarea medie a
unui film pe baza altor atribute, am ales să testăm diverse combinații de
parametri de intrare cu diferite modele de regresie. Pentru fiecare asociație de
parametri de intrare și model de regresie, am calculat indicele RMSE în scopul
determinării gradului de adecvare a
datelor la modelul respectiv. Rezultatele sunt prezentate, în ordinea
eficacității, în Tabela \ref{tab:models_performance_va}.

Parametrii de intrare sunt:

\begin{itemize}
  \item \textbf{G} - Genurile asociate filmului
  \item \textbf{K20} - Cuvintele-cheie (top 20)
  \item \textbf{K50} - Cuvintele-cheie (top 50)
  \item \textbf{K100} - Cuvintele-cheie (top 100)
  \item \textbf{P} - Popularitatea filmului
  \item \textbf{Vc} - Numărul de voturi ale filmului
\end{itemize}

Analiza efectuată a evidențiat că modelul de regresie Gradient Boosting
furnizează cele mai bune rezultate pentru fiecare combinație de parametri de
intrare utilizată. De asemenea, modelul XGBoost oferă rezultate competitive
pentru anumite combinații.

Iată clasamentul modelelor în funcție de parametrii de intrare și modelul de
regresie utilizat:

\begin{enumerate}
  \item \textbf{Gradient Boosting} - G, K100, P, Vc
  \item \textbf{Gradient Boosting} - G, K100, Vc
  \item \textbf{Gradient Boosting} - G, K50, Vc
  \item \textbf{XGBoost} - G, K100, Vc
  \item \textbf{Gradient Boosting} - G, K50, P, Vc
\end{enumerate}

\section{Constrângeri și direcții prospectabile de cercetare}

Pe parcursul desfășurării experimentelor am observat că popularitatea nu poate
fi anticipată cu certitudine, raportându-ne la parametrii pe care i-am utilizat,
însă aceasta se dovedește a fi utilă în previziunea votului mediu. De asemenea,
un număr semnificativ de coloane nu a fost luat în considerare, fapt care ar
putea justifica de ce nu am obținut rezultate relevante în ceea ce privește
prezicerea popularității filmelor. În plus, setul de date nu cuprinde bugetul
alocat fiecărui film. Bugetul și echipa din spatele fiecărui film constituie doi
factori esențiali în producerea filmelor de calitate.

În ceea ce privește direcțiile viitoare de cercetare, ne propunem să luăm în
considerare coloanele \textbf{cast} și \textbf{crew} pentru a încerca să
anticipăm popularitatea și votul mediu, întrucât echipa implicată în realizarea
fiecărui film reprezintă un factor determinant în calitatea și popularitatea
acestora.

\chapter{Concluzia}

Această lucrare a urmărit să înțeleagă și să analizeze în ce măsură anumite
caracteristici ale filmelor pot fi folosite pentru a prezice popularitatea sau
votul mediu al acestora. Setul de date folosit pentru această analiză a fost
TMDB 15000 Movies Dataset, un set extensiv de filme cu o serie de caracteristici
precum limba originală a filmului, popularitate, data lansării, votul mediu,
numărul de voturi, genuri, cuvinte cheie și membrii distribuției și echipei de
producție.

Rezultatele analizei noastre arată că prezicerea popularității filmelor este un
proces complex, care nu poate fi realizat cu un grad ridicat de precizie doar pe
baza caracteristicilor disponibile în setul nostru de date. Cu toate acestea, am
observat că votul mediu al filmelor poate fi prezis cu o mai mare acuratețe
folosind o combinație de caracteristici, incluzând genurile filmelor, cuvintele
cheie asociate, popularitatea și numărul de voturi. Aceste rezultate sugerează
că popularitatea și votul mediu al filmelor sunt determinate de o serie complexă
de factori, dintre care unii pot să nu fie disponibili în setul nostru de date.


\begin{table}
  \centering
  \renewcommand{\arraystretch}{2}
  \setlength{\arrayrulewidth}{0.5mm}
  \begin{tabularx}{\textwidth}{cccccc|ccccc}
    \multicolumn{6}{c|}{\textbf{Parametrii de intrare}} & \multicolumn{5}{c}{\textbf{Modele de regresie}} \\
    \cline{0-10}
    % Parameters.
    \rowcolor{gray!25}

        \multicolumn{1}{c}{\textbf{G}}
        & \multicolumn{1}{c}{\textbf{K20}}
        & \multicolumn{1}{c}{\textbf{K50}}
        & \multicolumn{1}{c}{\textbf{K100}}
        & \multicolumn{1}{c}{\textbf{Va}}
        & \multicolumn{1}{c|}{\textbf{Vc}}
        % Models.
        & \multicolumn{1}{c}{\textbf{L}}
        & \multicolumn{1}{c}{\textbf{DT}}
        & \multicolumn{1}{c}{\textbf{RF}}
        & \multicolumn{1}{c}{\textbf{GB}}
        & \multicolumn{1}{c}{\textbf{XGB}} \\
        \cline{0-10}

        X &  & X &  &  & X
        & 1.11761
        & 1.07040
        & 1.13712
        & 1.17415
        & 1.25839 \\

         &  & X &  & X & X
        & 1.11671
        & 1.26242
        & 1.12481
        & 1.10290
        & 1.11901 \\

        X &  & X &  &  &
        & 1.12039
        & 1.16990
        & 1.10402
        & 1.15218
        & 1.13616 \\

         &  &  & X & X & X
        & 1.11773
        & 1.44152
        & 1.15478
        & 1.10864
        & 1.11149 \\

         &  & X &  &  & X
        & 1.11671
        & 1.28159
        & 1.13687
        & 1.11159
        & 1.13244 \\

         & X &  &  & X & X
        & 1.11408
        & 1.12185
        & 1.12484
        & 1.11261
        & 1.12237 \\

         & X &  &  &  & X
        & 1.11409
        & 1.14186
        & 1.13502
        & 1.13274
        & 1.12592 \\

         & X &  &  & X &
        & 1.11664
        & 1.32812
        & 1.18910
        & 1.13383
        & 1.11459 \\

         &  &  & X &  & X
        & 1.11773
        & 1.38466
        & 1.14683
        & 1.13655
        & 1.11486 \\

        X &  &  &  & X & X
        & 1.11524
        & 1.18550
        & 1.16255
        & 1.30685
        & 1.36185 \\

  \end{tabularx}

  \caption{
    Performanta modelelor in a prezice popularitatea pentru diferite combinatii
    de parametri de intrare.}
  \label{tab:models_performance_p}
\end{table}


\begin{table}
  \centering
  \renewcommand{\arraystretch}{2}
  \setlength{\arrayrulewidth}{0.5mm}
  \begin{tabularx}{\textwidth}{cccccc|ccccc}
    \multicolumn{6}{c|}{\textbf{Parametrii de intrare}} & \multicolumn{5}{c}{\textbf{Modele de regresie}} \\
    \cline{0-10}
    % Parameters.
    \rowcolor{gray!25}

        \multicolumn{1}{c}{\textbf{G}}
        & \multicolumn{1}{c}{\textbf{K20}}
        & \multicolumn{1}{c}{\textbf{K50}}
        & \multicolumn{1}{c}{\textbf{K100}}
        & \multicolumn{1}{c}{\textbf{P}}
        & \multicolumn{1}{c|}{\textbf{Vc}}
        % Models.
        & \multicolumn{1}{c}{\textbf{L}}
        & \multicolumn{1}{c}{\textbf{DT}}
        & \multicolumn{1}{c}{\textbf{RF}}
        & \multicolumn{1}{c}{\textbf{GB}}
        & \multicolumn{1}{c}{\textbf{XGB}} \\
        \cline{0-10}

        X &  &  & X & X & X
        & 0.17701
        & 0.13005
        & 0.10593
        & \cellcolor{green!75} 0.09839
        & 0.10324 \\

        X &  &  & X &  & X
        & 0.17701
        & 0.11713
        & 0.10545
        & \cellcolor{green!50} 0.09840
        & \cellcolor{green!10} 0.09871 \\

        X &  & X &  &  & X
        & 0.17752
        & 0.11806
        & 0.10605
        & \cellcolor{green!30} 0.09865
        & 0.10019 \\

        X &  & X &  & X & X
        & 0.17752
        & 0.13337
        & 0.10649
        & 0.09873
        & 0.10422 \\

        X & X &  &  &  & X
        & 0.17772
        & 0.12021
        & 0.10689
        & 0.09885
        & 0.10027 \\

        X &  &  &  &  & X
        & 0.17833
        & 0.12032
        & 0.10818
        & 0.09891
        & 0.10146 \\

        X & X &  &  & X & X
        & 0.17772
        & 0.13389
        & 0.10687
        & 0.09892
        & 0.10449 \\

        X &  &  &  & X & X
        & 0.17833
        & 0.13562
        & 0.10783
        & 0.09893
        & 0.10613 \\

         &  &  & X & X & X
        & 0.18069
        & 0.13474
        & 0.11140
        & 0.10027
        & 0.10293 \\

         &  &  & X &  & X
        & 0.18069
        & 0.11130
        & 0.10308
        & 0.10036
        & 0.10094 \\

  \end{tabularx}

  \caption{
    Performanta modelelor in a prezice votul mediu pentru diferite combinatii de
    parametri de intrare.}
  \label{tab:models_performance_va}
\end{table}


\end{document}
