%
% report.tex
%

\documentclass[a4paper, 12pt]{report}

% UTF-8 characters.
\usepackage[utf8]{inputenc}

% Page format.
\usepackage[top=3cm, bottom=3cm, left = 2cm, right = 2cm]{geometry}

% 1.2 line spacing.
\usepackage{setspace}
\spacing{1.2}

% Romanian language support.
\usepackage{fontspec}
\usepackage{polyglossia}
\setdefaultlanguage{romanian}
\SetLanguageKeys{romanian}{indentfirst=true}
\usepackage{csquotes}
\DeclareQuoteStyle{romanian}
  {\quotedblbase}
  {\textquotedblright}
  {\guillemotleft}
  {\guillemotright}

% Embed images.
\usepackage{graphicx}
\graphicspath{./assets/}

% No page breaks between chapters.
\usepackage{etoolbox}
\makeatletter
\patchcmd{\chapter}{\if@openright\cleardoublepage\else\clearpage\fi}{}{}{}
\makeatother

% Chapter number and title on the same line.
\usepackage{titlesec}
\titleformat{\chapter}[hang]
{\normalfont\huge\bfseries}{\thechapter.}{0.5em}{}

% Abstract on the same page.
\renewenvironment{abstract}
{\small\quotation
{\bfseries\noindent{\large\abstractname}\par\nobreak\smallskip}}
{\endquotation}

% Metadata and title.
\usepackage{titling}
\title{Raport privind cercetarea și analiza unui set de date despre filme}
\author{
  Mathe Armand Cezar
  \hspace{2cm}
  Tritean Sergiu Daniel
}
\date{Iunie 2023}
\renewcommand{\maketitle}{
  \begin{titlepage}
    \begin{center}
      \large
      \textbf{Universitatea Babeș-Bolyai} \\
      \vspace{0.5cm}
      \textbf{Facultatea de Științe Economice și Gestiunea Afacerilor}\\
      \vspace{0.5cm}
      \textbf{Specializarea Informatică Economică}

      \vspace{7cm}

      \Huge
      \thetitle

      \vspace{6cm}

      \large
      \theauthor

      \vspace{3.7cm}

      \thedate
    \end{center}
  \end{titlepage}
}

\usepackage{lipsum}

% Document.
\begin{document}

\maketitle

\pagebreak

\tableofcontents

\pagebreak

\chapter{Introducere}

Industria cinematografică este una din cele mai influente și dinamice din lume.
Filmele nu numai că au o importanță culturală semnificativă, dar reprezintă și o
sursă importantă de venit. Predicția succesului unui film poate fi un instrument
valoros pentru producătorii de filme, permițându-le să aloce resurse în mod
eficient și să optimizeze strategiile de marketing. În această lucrare, încercăm
să răspundem la două întrebări fundamentale de cercetare:

\begin{enumerate}
  \item În ce măsură putem prezice popularitatea unui film pe baza altor
        caracteristici?
  \item În ce măsură putem prezice votul mediu al unui film pe baza altor
        caracteristici?
\end{enumerate}

Aceste întrebări sunt relevante deoarece filmele populare sau cu voturi medii
mari atrag multe vizionări, ceea ce poate genera venituri mai mari pentru
producătorii de filme. Lucrarea noastră vizează determinarea relațiilor dintre
caracteristicile de bază ale unui film și măsurătorile sale (vot mediu,
popularitate), cu scopul de a direcționa investițiile în domeniu spre proiectele
care ar putea avea un succes mai mare.

\chapter{Setul de date}

Setul nostru de date a fost inițial colectat prin scraping de pe site-ul TMDB
(The Movie Database), o resursă populară pentru informații despre filme și
seriale TV. Acest set de date a fost ulterior disponibilizat pe Kaggle, o
platformă pentru învățarea automată și analiza datelor, de către utilizatorul
XYZ care a efectuat procesul de colectare a datelor prin intermediul API-ului
TMDB.

Setul nostru de date include 15.000 de filme și conține o serie de
caracteristici pentru fiecare film, cum ar fi:

\begin{itemize}
  \item Titlul filmului: Numele oficial al filmului.
  \item Anul de lansare: Anul în care filmul a fost lansat.
  \item Genul: Genul sau genurile cărora le aparține filmul.
  \item Rating: Rating-ul mediu primit de film de la utilizatorii TMDB.
\end{itemize}

Și multe altele. Aceste informații oferă o imagine detaliată a fiecărui film și
permit aplicarea metodelor de învățare automată pentru a extrage înțelesuri
utile și a genera recomandări relevante.

\chapter{Rezultate si discuții}

\section{Analiza rezultatelor}

În această secțiune, vom examina rezultatele analizei noastre asupra setului de
date de filme. Vom aborda diferite metrici pentru a evalua performanța modelelor
noastre și a înțelege semnificația acestor rezultate.

\subsection{Interpretarea rezultatelor}

Aici vom interpreta și explica rezultatele pe care le-am obținut în urma
procesului de analiză. Vom încerca să identificăm modele, tendințe sau anomalii
care pot oferi o înțelegere mai profundă a datelor noastre.

\section{Implicații și discuții}

Această secțiune va oferi un spațiu pentru discuții mai generale legate de
rezultatele noastre. Vom încerca să evidențiem implicațiile acestora, precum și
orice limitări ale analizei noastre.

\subsection{Impactul asupra industriei cinematografice}

În acest subcapitol, ne vom concentra asupra implicațiilor potențiale ale
rezultatelor noastre asupra industriei cinematografice. În ce fel informațiile
pe care le-am descoperit pot influența deciziile viitoare în cadrul industriei?

\subsection{Limitări și viitoare direcții de cercetare}

În final, vom aborda limitările prezentei analize și vom propune potențiale
direcții pentru cercetările viitoare. Aceasta ne va permite să înțelegem mai
bine unde putem îmbunătăți și ce alte aspecte ar putea fi explorate în viitor.

\chapter{Concluzia}

Aceasta este secțiunea în care vom trage concluziile finale în urma analizei
noastre asupra setului de date de filme. Vom sublinia cele mai importante
descoperiri și vom discuta semnificația lor în contextul mai larg al industriei
cinematografice.

\end{document}
